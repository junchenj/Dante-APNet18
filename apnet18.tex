\documentclass{apnet18}

\usepackage{times}
\usepackage{epsfig}
\usepackage[TABBOTCAP]{subfigure}
\usepackage{tabularx}
\usepackage{graphicx}
\usepackage{color}
\usepackage{xspace}
\usepackage{thumbpdf}
\usepackage{listings}
\usepackage{verbatim}
\usepackage{hyperref}
\usepackage{booktabs}
\usepackage{colortbl}

\usepackage{colortbl,booktabs}%
\usepackage{subfigure}
\usepackage{amsmath}
\usepackage{algorithm}    
\usepackage{algorithmic} 



\hypersetup{pdfstartview=FitH,pdfpagelayout=SinglePage}

\setlength\paperheight {11in}
\setlength\paperwidth {8.5in}
\setlength{\textwidth}{7in}
\setlength{\textheight}{9.25in}
\setlength{\oddsidemargin}{-.25in}
\setlength{\evensidemargin}{-.25in}
%\setlength{\headsep}{0in}
%\pagenumbering{arabic}


\renewcommand\thesection{\arabic{section}} 





\begin{document}

\conferenceinfo{APNet 2018} {}
\CopyrightYear{2018}
\crdata{X}
\date{}

%%%%%%%%%%%% THIS IS WHERE WE PUT IN THE TITLE AND AUTHORS %%%%%%%%%%%%

\title{Dante: An FOV-Aware FEC-Based Multipath Protocol for 360-Degree Video Streaming}

\maketitle

%\thispagestyle{empty}


	
%%%%%%%%%%%%%  ABSTRACT GOES HERE %%%%%%%%%%%%%%
\section*{Abstract}

360-degree videos have been widely applied due to its unprecedented immersive experience. The appreciation of watching 360-degree videos
on untethered mobile devices, such as smartphone headsets,
is considered to be a more promising trend, which, however, is hampered by the
poor Quality of Experience (QoE), subject to limited bandwidth and error-prone characteristic of wireless networks.
Fortunately, the key observation that only a small portion of a 360-degree video is
perceived by users at any time, i.e., Field Of View (FOV), implies that unequal attention
should be given different regions spatially and can
be utilized to mitigate dependence on stable network condition and ultra-high bandwidth. Currently, the state-of-the-art schemes, like FOV-aware tile-based streaming in the application layer, almost based on FOV-aware bit-rate adaptation, transport protocol of which, however, fails to consider FOV. So, we propose an application-layer protocol, reliability scheme of which is FOV-aware and performs hierarchical protection to boost QoE of 360-degree videos in mobile scenarios. Experiments demonstrate our protocol achieves desirable improvements over the reference schemes in QoE of 360-degree videos.

\section{Introduction}
	
With the promise of immersive visual experience, 360-degree videos are widely applied in sports field, social field~\cite{facebook360}, and apps development field~\cite{google_developers}, etc. Meanwhile, it has been well deployed across mainstream content providers such as NBC (who broadcast the 2018 winter Olympics in VR), news outlets, such as CNN, New York Times, and user-generated content platform such as Youtube and Facebooks.
However, some significant challenges still remain as the major bottlenecks for 360-degree technologies, even worse in mobile scenarios.

The bottleneck can be summarized as that wireless links are characterized by limited resources and error-prone problems, while 360-degree video streaming
is characterized by bandwidth intensiveness and delay sensitivity.

%% One aspect of technological challenges stems from stringent delay and high bandwidth requirement. 
%% To prevent simulator sickness~\cite{Simulator_Sickness}, the whole system needs to react as fast as the refresh rate of Head-mounted Display(HMD), such as 120 Hz, in other words, this means application round-trip latency is supposed to be less than 10ms for imperceptible Motion-To-Photon(MTP) latency. Such the rigorous delay constraint prevents the implementation of those traditional protocols based on feedback and retransmissions.
For example, application round-trip latency is supposed to be less than 10ms for imperceptible Motion-To-Photon(MTP) latency in order to prevent simulator sickness~\cite{Simulator_Sickness}, due to the refresh rate of Head-mounted Display(HMD), such as 120 Hz. Such the rigorous delay constraint prevents the implementation of those traditional protocols based on feedback and retransmissions.

Meanwhile, 360-degree videos have the requirement of high bandwidth in order to enable immersive experience. The bit-rate of 8K 360-degree videos at 60 fps encoded using High Efficiency Video Coding (HEVC)~\cite{HEVC} is around 100 Mbps. However, according to OpenSignal~\cite{opensignal}, covering 77 countries in the world, almost a half of countries have access to 4G cellular network with only 10-25 Mbps speed in 2017. Obviously, it's not available and practical to deploy 360-degree videos service in mobile scenarios.    

Currently, based on the observation that only the FOV region of video is only perceived by user anytime, FOV-aware tile-based streaming~\cite{Viewport-adaptive}\cite{360ProbDASH}\cite{Adaptive_Streaming_Framework} \cite{Two-tier}\cite{Omnidirectional_Video_over_HTTP}\cite{Furion}, a scheme of application layer, is proposed to mitigate the requirement of bandwidth and delay, due to the introduction of FOV-aware bit-rate adaptation and video segment prefeatching. However, it achieves poor performance over wireless links, featured by limited bandwidth and error-prone characteristic, because the most advanced transport protocols it uses fail to consider FOV, for example, their reliability schemes of transport layer fail to consider FOV.     

In this paper, we propose Dante, as depicted in Figure 1, an application-layer 360-degree video protocol.
In this proposed Dante, instead of only transmissions, FEC is adopted to strengthen reliability scheme, which can recover data loss over wireless lossy links, mitigating video quality degradation. 
And the core of the reliability scheme is an FEC adaptation, which is FOV-aware and performs hierarchical error protection on different region of the video, spatially.

%We first design a FOV-aware video distortion model, and from the perspective of reliability schemes, design an FOV-aware FEC parameter adjusting algorithm based on that model to achieve low latency. Furthermore, we design an FOV-aware packet scheduling algorithm, which preferentially allocates better bandwidth for more crucial data, and thus boosting video quality and achieving graceful degradation even under poor network condition.

\begin{figure}[ht]
	\centering
	\includegraphics[scale=0.4]{paper_figs/stack_dante.png}
	\caption{Illustration Of Dante}
	\label{paper_figs:pathdemo}
\end{figure}

\begin{figure}[ht]
	\centering
	\includegraphics[scale=0.25]{paper_figs/tradeoff_only_goodput.png}
	\caption{Illustration Of Dante}
	\label{paper_figs:pathdemo}
\end{figure}



\section{Background And Motivation}

%\subsection{360-degree Tile-based Streaming}

%%In order to mitigate dependence on high bandwidth, FOV-aware tile-based streaming scheme, based on Dynamic Adaptive Streaming over HTTP (DASH) \cite{MPEG-DASH}, is extensively studied in recent years. For most typical works~\cite{Viewport-adaptive}\cite{360ProbDASH}\cite{Adaptive_Streaming_Framework} \cite{Two-tier}\cite{Omnidirectional_Video_over_HTTP}\cite{Furion}, 360-degree videos are split into segments of equal length, such as 1s, and the server offers multiple bit-rate of representations for every segment. Furthermore, with the technology of motion-constraint tile sets (MCTS), every segment can be encoded into multiple tiles spatially, as depicted in Figure 1, each of which can be independently decoded, stored into a single file and sent to clients alone. So, the server offers representations that also differ spatially by having a Quality Emphasized Region (QER)~\cite{Viewport-adaptive}: a region of the video which is made up of tiles with higher bit-rate than the rest of tile of the remaining of the video. Clients periodically pre-fetch a representation for the next segment such that the bit-rate adapts to available bandwidth and QER best matches the expected viewport of users.
%Unfortunately, these state-of-the-art works~\cite{360ProbDASH} \cite{Adaptive_Streaming_Framework} \cite{Two-tier} \cite{Omnidirectional_Video_over_HTTP} \cite{Furion}, which focus on the delivery of 360-degree video and only solved the problem of FOV-aware bit-rate adaptation in application layer, failed to design a effective FOV-aware scheme to guide the transport protocol to counter limited bandwidth and time-varying problem in mobile scenarios. 
%For example, they still use HTTP, built on top of TCP, as their transmission protocol. which, however, coupling flow and congestion control, suffers from poor throughput and delay performance in wireless networks featured by high loss rate, is inappropriate for 360-degree video delivery.

% Unfortunately, they only strive to optimize FOV-aware bit-rate adaptation schemes in application layer and failed to consider a effective FOV-aware scheme to guide the transport protocol. Thus, they can not perform well over lossy links, even worse in mobile scenarios.  

%However, even combined with 360-degree tile-based streaming, due to the failure to consider FOV, 

%\subsection{The introduction of FEC}
%	 MultiPath Parallel Transmission, considering mobile devices, like smartphone, almost equipped with diffrent radio interfaces (eg.Wi-Fi and LTE), is considered to be a promising way to solve the problem of limited bandwidth over wireless links. IETF-MPTCP \cite{IETF-MPTCP} is proposed and suffers from the performance degradation subject to the bottleneck link. These works~\cite{MPLOT}\cite{FMTCP} \cite{HMTP}, due to the introduction of FEC and well designed data allocation algorithm, not only aggregate capacities across paths but counter wireless network's time-varying characteristic, mitigating the head-of-line blocking and packet out-of-order in multiple diverse network.



	Generally, VR headset has a 110 degree horizontal FOV and a 90 degree vertical FOV, and the fraction of videos of video which extracted and display on HMD would be $\frac{{110^\circ }}{{360^\circ }} \times \frac{{90^\circ }}{{180^\circ }}{\rm{ = }}15{\rm{\% }}$. Obvious, the data of FOV region, which is be viewed by users, is more imporatant than non-FOV data.
	From the perspective of application layer, the FOV-aware tile-based streaming schemes, based on Dynamic Adaptive Streaming over HTTP (DASH), suggests that, 360-degree videos are split and encoded into multiple tiles spatially, each of which can be independently decoded, stored into a single file. Furthermore, according to the distance from the expected viewpoint of users, 360 videos would be split into two or three regions spatially, each of which would be composed of multiple contiguous tiles and store into a single file, as depicted in Figure 3.
	
	\begin{figure}[ht]
		\centering
		\includegraphics[scale=0.2]{paper_figs/tileSplit.png}
		\caption{360-degree Video Representation}
		\label{paper_figs:pathdemo}
	\end{figure}	
	
	
	However, the state-of-the-art transport schemes, which are not FOV-aware, spend as same amount of bandwidth on reliable delivery of trivial data, i.e., non-FOV data, as FOV data. So, can we, from the perspective of reliability of transport, design a protocol, which prioritizes the reliability of FOV data over non-FOV data. Thus, it is supposed to boosts the whole system performance by sacrificing some degree of quality of non-FOV data.

	Instead of only retransmissions, FEC\footnote{In Dante, FEC is generated through erasure coding of blocks of packets, to recover lost packets, which should be distinguished from FEC, computed on individual packets using channel coding, recovers from bit errors.} is adopted to be a proactive scheme of reliability in our proposed protocol. It can achieve low delay by mitigating retransmissions and is generally be in favoured by real-time video services. However, how to adjust the degree of FEC redundancy is a problem. 
	 
	For example, considering a chunk of data organized into K packets, with equal length, the FEC encoder takes the K data packets, and adds redundant M FEC packets to create a coded block of size ${\rm{B = (K + M)}}$. The receiver can recover completely the origin data of K packets if any at least K packets of ${K + M}$ packets are received. The code rate is equal to ${K/(K + M)}$ and the redundancy of FEC is ${(M/K)}$ . Obviously, it can recover M packets of loss over lossy links at most and when FEC redundancy packets is more, the coding system has more powerful recoverability. 	 
	
	
    Increasing FEC redundancy can improve recovery probability of data packets in order to mitigate the degradation of video quality caused by packet transmission loss. However, with the increasing of redundancy and computing overhead of coding \cite{ASCOT}, the over-provisioning of redundancy may enlarge end-to-end delay, even causing unnecessary data loss caused by the hit of deadline. As depicted in Figure 2, with the increment of FEC redundancy, while the original data can be recovered with higher probabilities, the goodput first goes up to a peak point and degrades  latter, gradually. So, it's important to carefully adjust the redundancy of FEC in order to balance the tradeoff between recovery probability and goodput performance.  
	 
	We consider the situation where the video is split into two region, FOV region and non-FOV region. We find, compared to the schemes in which different regions are encoded with same degree of FEC redundancy, it can result in better video quality that different regions of video segment are encoded with different redundancy of FEC before the video segment is sent to client. 
	 
    For example, the network status is set where average packet loss rate is set to 1\%, bandwidth is set to 40Mb/s. Meanwhile, the length of video segment is equal to o.5s, the size of FOV region is selected into 4Mb, the size of non-FOV region is 20Mb and playback deadline is set to 500ms. As illustrated in Table 1, while the set where all regions is encoded wih the same redundancy, perform worser among them the combination, the redundancy of FOV region and non-FOV is 2\% and 5\%, respectively, achieves obvious upgrade of video quality in FOV, and little degradation of video quality in non-FOV. Furthermore, FOV data, which is more likely perceived by users, is more important to QoE of video than non-FOV and the whole video quality is boost. So, when the network is limited-source and error-prone, if carefully designed, the schemes in which systems prioritized FOV data over non-FOV data, i.e., reliability VS best-effort, can effectively boost the whole system performance.       
	
	\begin{table}
		\centering 
		\scriptsize
		\begin{tabular}{p{2.0cm}p{2.0cm}p{1.6cm}p{1.6cm}}
			\rowcolor[gray]{0.9} 
			\hline
%			 The length of segment &The size of QER region(Mb) & The size of non-QER region(Mb) &
			The FOV redundancy(\%) &The non-FOV redundancy(\%) & Expected PSNR of FOV(dB) & Expected PSNR of non-FOV(dB)\\
			\hline
			3  &  3  &  35  &  35\\    
			\hline
			2  &  5  &  43  &  33\\ 
			\hline
			
		\end{tabular}
		\caption{The Video Quality of different FEC redundancy combinations}
		\label{}
	\end{table}
	 
	 
	Based on the above observations, unlike the 360-degree tile-based streaming protocols, which perform FOV-aware bit-rate adaptation on different region, we proposed Dante, reliability scheme of which performs on different region of videos in a hierarchical fashion, i.e, preferentially provisioning the data closer to FOV with more FEC redundancy. Thus, the data, which more strongly affects QoE of video, transmitted over lossy links, is supposed to integrally received with a higher probability and thus QoE of 360-degree video can be boosted notably.
 
%Manwhile, in Table 1, we summarize the main differences of Dante with the existing multipath schemes. 
To the best of our knowledge, Dante is the first FOV-aware 360-degree video protocol over heterogeneous wireless networks.



\section{Protocol Design}
%Unlike the video bit-rate adaptation of FOV-aware streaming, Dante, from the perspective of reliability scheme, preferentially provisions the tiles, viewed by user with higher probabilities and more important to video quality, with more FEC redundancy.  



%We consider the expected video quality loss caused by transmission loss and decoding dependencies of video codec, both of which depend on the FEC parameter adjusting procedure.	
%The core of our system is the FEC adaptation scheme, which is supposed to allocate more redundancy to the data, more closer to the FOV region in order to achieve good video quality as much as possible. But, how to decide which FEC redundancy is optimal is a problem?




\subsection{FEC Redundancy Adaptive Adjusting}
\subsubsection{Optimization Problem}

Obviously, The aim of FEC redundancy adaptation is to minimize the distortion of video viewed by users.
Hence, the problem of FEC redundancy adaptation can be formulated into the following expression of optimization problem.
Given estimated network parameter, such as estimated packet loss rate $\Pi _m^\alpha$, RTT, available bandwidth, the corresponding optimization problem can be formulated as:
\begin{eqnarray}
&{\{ R_m^a\} _{1 \le m \le M,a \in Q}} = \arg \min (\sum\limits_{i = 1}^M {d_{m,effective}}).  \\
&{\rm{subject}}{~~\rm{to}}{~~~T^{tran}} \le {T_{GOP}}~~~~~~~~~~~~~~~~~, \\
&{\rm{and}}{~~~~~~\lambda ^p}(\Phi ) \le {\mu _p}\begin{array}{*{20}{c}}
{}
\end{array}{\rm{}},{\rm{}}~~~~~~~1 \le p \le P~~~,\\
&\begin{array}{*{20}{c}}
{{T^{tran}}{\rm{ = }}}&{\frac{{\sum\nolimits_{m = 1}^M {\sum\nolimits_{\alpha  \in Q} {(V_m^\alpha  \cdot (1 + R_m^a))} } }}{{t{r^{TFRC}}}}}
\end{array},\\
&{\lambda ^p}(\Phi ) = \lambda \frac{{\sum\nolimits_{m = 1}^M {\sum\nolimits_{\alpha  \in Q} {{{(V_m^\alpha  \cdot (1 + R_m^a))}_{_{\left\{ {\Phi _m^\alpha  \in p} \right\}}}}} } }}{{\sum\nolimits_{m = 1}^M {\sum\nolimits_{\alpha  \in Q} {V_m^\alpha } } }}.
\end{eqnarray}

where, the objective function is the minimization of the sum of video distortion for a GOP, subject to constraints of both deadline and available bandwidth. We assume that, the data, from any layer of any frame, makes up an FEC block. The decision variable, \ie output, $\{ R_m^\alpha \}$, is the FEC redundancy of the block for the $\alpha$ layer in the m-th frame. The first constraint (Eq. (3)) indicates that, due to timeliness of video, the data of every GOP should be delivered to the client side before the delay constraint $T_{GOP}$, \ie the transmission time of GOP $T^{tran}$ should be not greater than $T_{GOP}$.
Furthermore, ${V_m^\alpha }$ denotes the size of the $\alpha$ layer for the m-th frame. So, given the packet size, S, the estimated round trip time, RTT and estimated packet loss rate, $\Pi_m^\alpha$, the transmission time of GOP, $T^{tran}$ can be calculated by the size of GOP being divided by the transmission rate, $t{r^{TFRC}}$, according to \cite{TRFC}. 
As for the second constraint, (Eq. (4)) represents the video traffic rate, considering the introduction of FEC redundancy, is supposed to be not greater than available bandwidth.

Therefore, FEC redundancy adaptation can be converted to solving the above optimization problem.

Naturally, we can imagine that, given netowrk parameters, such as estimated packet loss rate, and FEC redundancy, if system of sender side can beforehand calculate quantitatively the quality of video viewed by users of the receiver side, in a somewhat fashion, the FEC redundancy bringing greatest video quality can be a good one.

Fortunately, video distortion computing model can be used to accomplish this. PSNR is used to evaluate video distortion which is calculated via Mean Squared Error (MSE). So, we directly use MSE to denote the video distortion.
In the next, we introduce video distortion computing model, as well as how to beforehand calculate the video distortion.

\subsubsection{The Analysis of Objective Function}
In this section, we focus two things, this first is that, given estimated packet loss rate, how the distortion computing model manifests quantitatively the effect which different FEC redundancy has on the whole distortion?. 
On that basis, the other things is that, since the FOV region which users request can not best completely match the FOV which users finally watch, we need to calculate the quality of video, actually viewed by users, by calculating the expected value of video distortion. 
We next introduce how to mathematically express the the expected value of video distortion. 

According to \cite{distortion_model} and \cite{CMT-VR}, given estimated packet loss rate $\pi _{m}^t$ and FEC redundancy $R_m$,
the distortion of the m-th frame for every video GOP (group of pictures) can be formulated as:${d_m} = d_{m,trunc}(\pi _{m}^t, R_m) + {d_{m,drift}}$. $d_{m,trunc}(\pi _{m}^t, R_m)$ denotes the portion of distortion caused by transmission loss and overdue. ${d_{m,drift}}$ denotes the distortion caused by decoding dependence among frames of video codec. We mainly focus on $d_{m,trunc}(\pi _{m}^t, R_m)$, which is directly related to FEC redundancy.

However, unlike non-360-degree videos, only a small portion of 360-degree videos spatially is perceived by users anytime. Furthermore, according to 360ProbDASH\cite{360ProbDASH}, each tile of 360-degree videos requested by users, is expected to be watched by users with a probability and the probability follows Gaussian Distribution at any time. So we customize the traditional distortion model into the expected value of distortion, called as the effective distortion, which can be calculated via the sum of the distortion of each region being multiplied by its probability of viewing. 
As a result, given estimated packet loss rate $\pi _{m,\alpha }^t$ and FEC redundancy $R_m^\alpha$, each video frame, the effective distortion is formulated as:
\begin{equation}
{d_{m,effective}} = \sum\limits_{\alpha  \in Q} {{\gamma ^\alpha }(d_{m,trunc}^\alpha (\pi _{m,\alpha }^t, R_m^\alpha) + d_{_{m,drift}}^{^\alpha })}
\end{equation}
where Q denotes the layer set of 360-degree videos, which includes FOV layer, cushion layer and outmost layer, as depicted in Figure 3. And given $\alpha $
layer, ${\gamma ^\alpha }$ denotes the accumulated viewing probability of all tiles in the $\alpha$ layer, formulated as:
\begin{equation}
{\gamma ^\alpha } = \sum\limits_{i = 1}^{{\Omega ^\alpha }} {{p_i} \cdot
	{S_i}}
\end{equation}
where ${p_i}$ stands for viewing probability of the i-th tile in the $\alpha $
layer, ${S_i}$ denotes area of the i-th tile and
${\Omega ^\alpha }$ denotes tiles set of the corresponding layer. 

Obvious, the tiles of FOV, requested by users and viewed by users with higher probabilities, are attached with greater weights, \ie accumulated probability, than non-FOV. Thus, improving the distortion of the FOV region can bring more performance gain in video distortion than non-FOV region. 

Meanwhile, in Eq. (7), given the estimated packet loss rate $\pi _{m,\alpha }^t$ and FEC-redundancy rate $R_m^\alpha$, $d_{m,trunc}^\alpha(\pi _{m,\alpha }^t, R_m^\alpha)$ denotes the expected value of MSE for the $\alpha$ layer of the m-th frame, which is
formulated as:
\[d_{m,trunc}^\alpha (\pi _{m,\alpha }^t, R_m^\alpha) = \widehat \delta _m^\alpha  + \Pi _m^\alpha (\pi _{m,\alpha }^t, R_m^\alpha )\cdot\delta _m^\alpha ,1 \le m \le M\]	

where, $d_{m,trunc}^\alpha (\pi _{m,\alpha }^t, R_m^\alpha)$ is proportional to $\Pi _m^\alpha (\pi _{m,\alpha }^t, R_m^\alpha )$. And given a frame $m$, region $\alpha$, FEC-redundancy rate $R_m^\alpha$, and estimated packet loss rate $\pi _{m,\alpha }^t$,
$\Pi _m^\alpha (\pi _{m,\alpha }^t, R_m^\alpha )$ denotes the effective data loss rate, \ie the percentage of lost symbols for the $\alpha$ layer of the m-th frame, caused by transmission loss and expired arrival after the introduction of FEC redundancy,  which is
formulated as:
\begin{small}
\begin{eqnarray}
&\Pi _m^\alpha(\pi _{m,\alpha }^t,R_m^\alpha)  = \left\{ {\begin{array}{*{20}{l}}
{0,\begin{array}{*{20}{c}}
{}
\end{array}if\begin{array}{*{20}{c}}
{\pi _{m,\alpha }^t + (1 - \pi _{m,\alpha }^t)}
\end{array}\cdot\pi _{m,\alpha }^o < \frac{{n - k}}{n},}\\
{\begin{array}{*{20}{c}}
{\pi _{m,\alpha }^t + (1 - \pi _{m,\alpha }^t)}
\end{array} \cdot \pi _{m,\alpha }^o,\begin{array}{*{20}{c}}
{}
\end{array}{\rm{otherwise}}.}
\end{array}} \right.
\end{eqnarray}
\end{small}

where given the $\alpha$ layer of the m-th frames, $(n-k)$ denotes the number of allocated FEC repair packets, $\frac{{n - k}}{n}$ stands for tolerant packet loss rate and $\pi _{m,\alpha }^o$ denotes the overdue loss rate. Obviously, $\Pi _m^\alpha$ is equal to 0 if the provisioning of redundant repair packets is sufficient for countering packet drops caused by transmission loss and expired arrival. 



\subsubsection{An Algorithm To Solve The Optimal Problem}
According to (Eq. 8), the derivation procedure of optimal FEC redundancy can be thought of as the procedure that the system make the effective data loss rate $\Pi _m^\alpha(\pi _{m,\alpha }^t, R_m^\alpha )$ approach the tolerant loss rate $\frac{{n - k}}{n}$.  

Obviously, this require that every FEC repair packet is carefully allocated to each layer of all frames in order to minimize the gap between the $\frac{{n - k}}{n}$ and $\Pi _m^\alpha$. 

we first obtain the mean redundancy of the data in GOP, $R$, from (Eq.2) and (Eq. 3) in order to meet the delay constraint and available bandwidth constraint. And the total number of FEC repair packets, denoted with $A$, is derived via the number of original packets K multiplied by R. Then all the repair packets for one GOP should be allocated into different layers of all frames in the GOP. 
Essentially, this problem is a discrete optimization problem.
It's computational prohibitive to apply the exhaustive search, which should consider all the possible combination of FEC redundancy for all layer of all frame, for the global optimal solution. 
To solve this problem, we design a fast research algorithm, to obtain a sub-optimal solution of FEC redundancy adaptive problem, shown in Algorithm 1. 

\begin{algorithm}[!h] 
	\scriptsize
	\centering 
	\caption{FEC redundancy adaptative algorithm}%算法标题      
	\begin{algorithmic}[1]%一行一个标行号
		\STATE $R = \min (Eq.~(3), Eq.~(4))$ , according to delay constraints, Eq. (3) and
		bandwidth constraints Eq. (4),
		
		\STATE $A = \frac{V}{S} \cdot R$ 
		
		\FOR{$\alpha  \in Q$}  
		
		\STATE Calculate $\gamma ^\alpha$ , according to Eq. (1),
		
		\ENDFOR		
		
		\FOR{$i = 1{\rm{ }} to {\rm{ }}A$}
		\STATE $index{\rm{ }} = {\rm{ }}0,{\rm{ }}{\Delta _d}{\rm{ }} = 0$
		\FOR{$m = 1{\rm{ }}to{\rm{ }}M$}
		\FOR{$\alpha  \in Q$}
		
		\STATE ${d_{effective}} = \sum\limits_{0 \le m \le M} {\sum\limits_{\alpha 
				\in Q} {{\gamma ^\alpha }(d_{_{m,trunc}}^{^\alpha } + d_{_{m,drift}}^{^\alpha
				})} } $
		\STATE ${A_{m,\alpha }} = {A_{m,\alpha }} + 1$
		\STATE $\Delta  = \left| { - {d_{effective}} + \sum\limits_{0 \le m \le M}
			{\sum\limits_{\alpha  \in Q} {{\gamma ^\alpha }(d_{_{m,trunc}}^{^\alpha } +
					d_{_{m,drift}}^{^\alpha })} } } \right|$
		\STATE ${A_{m,\alpha }} = {A_{m,\alpha }} - 1$
		\IF{$\Delta  \ge {\Delta _d}{\rm{ }}$}
		\STATE $index{\rm{ }} = m,\begin{array}{*{20}{c}}
		{layer}
		\end{array} = \alpha ,{\Delta _d} = \Delta$
		\ENDIF
		\ENDFOR 
		\ENDFOR
		\STATE ${A_{index, layer }} = {A_{index, layer }} + 1$
		\ENDFOR
		\RETURN ${\left\{ {{R_{m,\alpha }} = \frac{{{A_{m,\alpha }}}}{{{K_{m,\alpha }}}}} \right\}_{(1 \le m \le M,\alpha  \in Q)}}$
	\end{algorithmic}  
\end{algorithm} 

In Algorithm 1, the aim of line 10 to line 14 is to compare the whole GOP distortion gain after the allocating of one repair packet to one of all possible blocks, which include all layer of all frame in one GOP.  That procedure obtains a local optimal solution each time and as a result, the number of repair packet of the data block, corresponding to the greatest degradation of video distortion, is supposed to be plus one. The complexity of algorithm is $O(N \cdot A \cdot Q)$, and Q is generally equal to 2 or 3.


\subsection{System Overview}

Dante is proposed to support high-quality 360-degree video
streaming service over the wireless network. In Dante, UDP combined with the systematic FEC, RS code, is integrated to provide data delivery service over wireless networks. And only the data of I frames is retransmitted if no ACK is received by the server in time ${T^I}$, in order to guarantee the video data received is decodable by video codec.
Besides, TCP is supplementary to exchange control information, which is of significance. The overall protocol architecture is illustrated in Figure 2.

\begin{figure*}[ht]
	\centering
	\includegraphics[scale=0.4]{paper_figs/architecture_simple_0624_v7.png}
	\caption{The Architecture of Protocol}
	\label{paper_figs:pathdemo}
\end{figure*}



%*****Instantaneous PSNR In relatively Bad Network Condition*****	

\begin{figure*}[!t]
	%\begin{figure*}[t]
	\centering
    % \subfigure[Video Sequence 1]  {\includegraphics[scale=0.26,angle=0]{paper_figs/evaluation_result/sub/ins_psnr_v2_good.png}}
    % \subfigure[Video Sequence 2]  {\includegraphics[scale=0.26,angle=0]{paper_figs/evaluation_result/sub/ins_psnr_v3_good.png}}
    \subfigure[Video Sequence 1]  {\includegraphics[width=0.5\textwidth]{paper_figs/dante-good-video-1.pdf}}
    \hspace{-0.3cm}
	\subfigure[Video Sequence 2]  {\includegraphics[width=0.5\textwidth]{paper_figs/dante-good-video-2.pdf}}
	\vspace{-0.3cm}
	\caption{Instantaneous PSNR In Relatively Good Network Condition}
	\vspace{-0.4cm}
	\label{fig:apuct}
\end{figure*}

%*****Instantaneous PSNR In relatively Good Network Condition*****
\begin{figure*}[!t]
	%\begin{figure*}[t]
	\centering
% 	\subfigure[Video Sequence 1]  {\includegraphics[scale=0.275,angle=0]{paper_figs/evaluation_result/sub/ins_psnr_v1_bad.png}}
% 	\subfigure[Video Sequence 2]  {\includegraphics[scale=0.27,angle=0]{paper_figs/evaluation_result/sub/ins_psnr_v2_bad.png}}
    \subfigure[Video Sequence 1]  {\includegraphics[width=0.5\textwidth]{paper_figs/dante-bad-video-1.pdf}}\hspace{-0.3cm}
	\subfigure[Video Sequence 2]  {\includegraphics[width=0.5\textwidth]{paper_figs/dante-bad-video-2.pdf}}
	\vspace{-0.3cm}
	\caption{Instantaneous PSNR In Relatively Bad Network Condition}
	\vspace{-0.4cm}
	\label{fig:apuct}
\end{figure*}	



\section{Performance Evaluation}




\subsection{Reference schemes}
We use two 360-degree video sequences downloaded from Youtube as our dataset to 
evaluate Dante and compare it with TCP-based 360-degree streaming 
protocols, FOV-aware DASH \cite{Omnidirectional_Video_over_HTTP} and traditional FEC-enabled streaming protocols MTMTP \cite{MPMTP} and CMT-VR \cite{CMT-VR}, both of which are not FOV-aware. 

%We evaluate the performance of Dante to compare with two video transport schemes, MPMTP~\cite{MPMTP} and CMT-VR~\cite{CMT-VR}, both of which are not FOV-aware and also utilize FEC to mitigating unnecessary retransmission.

\begin{packeditemize}

\item {\em MPMTP\cite{MPMTP}:} .
This video protocol utilizes the FEC to recover the data, completely abandoning the retransmission,  in order to maximize the received data rate to prevent playback buffer starvation. This scheme performs in a video content-agnostic fashion. 

\item {\em CMT-VR\cite{CMT-VR}:}
CMT-VR utilize a quality-driven FEC redundancy allocation to minimize the distortion of each GOP and Raptor code is used to mitigate retransmission. While this scheme considers frame priority, \ie I, P frames, it is in a non-FOV-aware fashion. 

\item This work\cite{Omnidirectional_Video_over_HTTP} utilize an FOV-aware tile-based streaming scheme to mitigate the cost of bandwidth. It's built on the HTTP, using TCP as underlying transport protocol.

\end{packeditemize}


\subsection{Experimental Set-up}
\textbf{Experiment topology:} The server is connected with the client through one lossy links and Dante is deployed on both server and client side. The topology is not shown because of space limitation. The experiment scenario is that sources send the data to sinks through one lossy links with the request of video data.

\textbf{Testbed configuration:} The sources and sinks are commodity servers with Ubuntu 16.4 (kernel 4.40), each of which is equipped with an Intel(R) Core(TM) i3-4150k cpu @ 3.5GHz (4 cores), one Intel 82599ES 10G dual port NICs and 32 G memory.

\begin{table}
	\centering 
	\scriptsize
	\begin{tabular}{cp{1.5cm}p{1.0cm}|p{1.5cm}p{1cm}}
%		\rowcolor[gray]{0.9} 
		\hline
		 & $T_I$ &  200ms   &   The length of segment  & 1s \\
        \hline
		&  receiver buffer size  &  100Mb  &  The size of GOP  &  15 \\
		\hline

	\end{tabular}
	\caption{Parameter set
	\label{}
\end{table}




\textbf{Parameter sets}

Timeout for triggering I frames' retransmission, $T_I$, is set to 200ms, the size of GOP is set to 15 and the length of video segment, which is request by users each time, is sent to 1s, equal to the time of 2 GOPs. Meanwhile, the size of the receiver buffer is set to 100Mb.
We evaluate the quality of video by computing the expected value of PSNR. So, given a video segment, it's PSNR can be calculated via: \[{V_{PSNR}} = \sum\limits_{i = 1}^{{\Omega ^\alpha }} {({\gamma ^\alpha } \cdot V_{PSNR}^{^\alpha })} \]
where, ${V_{PSNR}^{^\alpha }}$ is the PSNR of the layer $\alpha$. According to a tiles viewing probabilities \cite{360ProbDASH} and (Eq. 7), $\gamma ^\alpha$ can be approximately obtained and $\gamma ^\alpha$ of FOV layer, cushion layer and outmost layer is almost 0.6, 0.3, 0.1, respectively.

\textbf{Network parameter set:} Gilbert model is adopted to mimic the packet loss pattern in real wireless networks, supported by traffic control (TC)~\cite{TC}, in which four parameters($\xi _i^G$, $\xi _i^B$, 1-h and 1-k) are needed, $\xi _i^G$ and $\xi _i^B$ are transition probabilities between the bad and good state, 1-h and 1-k is the loss probability in the bad state and good state, respectively. In our testbed, 1-h and 1-k are set as 1 and 0, respectively. Meanwhile, average packet loss rate is equal to $\pi _i^B = \xi _i^B/(\xi _i^B\\ + \xi _i^P)$. And the bandwidth is also set by TC. The detailed Parameter set is seen in Table 2. 

\begin{table}
	\centering 
	\scriptsize
	\begin{tabular}{cp{1.0cm}p{1.6cm}p{0.8cm}p{2.3cm}}
		\rowcolor[gray]{0.9} 
		\hline
		(A)  &  Time(Sec.)    & network capacity(Mbps)       &  RTT(ms) &  Average Pakcet loss rate(\%) \\

		
		&  0${\sim}$60   &  30         &    50    &  0.5 \\

		\hline
		\rowcolor[gray]{0.9}
		\hline
		(B)  &   Time(Sec.)   & Bandwidth(Mbps)       &  RTT(ms) &     Average Pakcet loss rate(\%)  \\
		
		&  0${\sim}$60   &  20         &    50    &  1.5\\
		
		\hline
		
	\end{tabular}
	\caption{Network Condition of Two Wireless networks: (A)Relatively Good Wireless Conditions And (B)Relatively Bad Wireless Conditions}
	\label{}
\end{table}


% \begin{figure*}[!t]
% 	%\begin{figure*}[t]
% 	\centering
% 	\subfigure[Video Sequence 1]  {\includegraphics[scale=0.26,angle=0]{paper_figs/Dante_VS_TCP_1.png}}
% 	\subfigure[Video Sequence 2]  {\includegraphics[scale=0.26,angle=0]{paper_figs/Dante_VS_TCP_2.png}}
% 	\vspace{-0.3cm}
% 	\caption{The PSNR of Dante_VS_FOV-Aware_DASH}
% 	\vspace{-0.4cm}
% 	\label{fig:apuct}
% \end{figure*}


\subsection{Performance Comparison With Existing Protocols}

Then, Figure 5 and Figure 6 compare instantaneous PSNR of three video sequences in good network condition and bad network condition, respectively. The result shows that Dante achieves 20\% to 30\% 360-degree video PSNR performance gain, compared to MPMTP and CMT-VR. The reason why MPMTP performs worst in all protocols is that, despite no involving retransmission and maximizing the throughput, it doesn't consider video's inherent feature, such as decoding dependencies of video codec, \ie I frames's importance, and thus fails to utilize effectively the network allocation to boost video quality. Meanwhile, CMT-VR performs better than MPMTP, due to its consideration of frame priority. However, unfortunately, non-FOV-aware reliability scheme makes it waste valuable bandwidth on trivial data, thus CMT-VR is the secondary one. Dante takes into account not only traditional video features aforementioned, but FOV. Benefiting from the hierarchical protection spatially and temporally, Dante achieves desirable upgrade in instantaneous PSNR. Meanwhile, we find, compared to TCP without FEC, the average CPU time of Dante's, including the cpu time of receiving socket, FEC decoding, video codec decoding as well as video rendering, due to the introduction of FEC computing overhead, increases from 114\% to 125\% for the sender side, and from 113\% to 126\% for the receiver side. The almost 11.5 percentage of the extra cpu overhead of FEC can be ignored considering the performance gain Dante achieves.    




\section{Conclusion}
In this paper, we propose a multipath protocol for 360-degree videos, which, based on multipath and FEC, performs hierarchical protection to counter limted bandwidth and error-prone problems in wireless networks. Experiments demonstrate Dante achieves desirable gains on 360-degree video QoE against reference schemes.  

\section*{ACKNOWLEDGEMENT}
The work was supported by the National Key Basic Research Program of China (973 program) under Grant 2014CB347800, National Key Research and Development Program of China under Grant 2016YFB1000200 and the National Natural Science Foundation of China under Grant No. 61522205, No. 61772305, No. 61432002. Dan Li is the corresponding author of this paper.

\bibliographystyle{abbrv}
\begin{small}
	\bibliography{apnet18}
\end{small}
\label{last-page}

\end{document}






\begin{figure}[ht]
	\centering
	\includegraphics[scale=0.2]{paper_figs/RS_tradeoff.png}
	\caption{Tradeoff between recoverability and goodput}
	\label{paper_figs:pathdemo}
\end{figure}	 

\begin{figure}[ht]
	\centering
	\includegraphics[scale=0.2]{paper_figs/RS_tradeoff_a.png}
	\caption{Tradeoff between recoverability and goodput}
	\label{paper_figs:pathdemo}
\end{figure}	




\begin{table*}
	\centering 
	\scriptsize
	\begin{tabular}{||p{1.35cm}<{\centering}||p{1.5cm}<{\centering}||p{1.7cm}<{\centering}
			||p{1.65cm}<{\centering} ||p{0.85cm}<{\centering} ||p{2.65cm}<{\centering}
			||p{2.0cm}<{\centering} || p{0.85cm}<{\centering} || p{0.85cm}<{\centering}||}
		\hline
		\rowcolor[gray]{0.9}
		\hline
		Solution & protocol layer   & video distortion model & data recovery& adaptive FEC & FEC parameter decision& data
		allocation&video-awareness& FOV-awareness\\
		\hline
		\hline
		MPLOT\cite{MPLOT} & transport layer & $\times$ & RS codes \& retransmissions & \checkmark
		& balance between goodput and recovery probability & packet generation's order & $\times$ & $\times$ \\
		\hline
		FMTCP\cite{FMTCP}    & transport layer  & $\times$ &Raptor codes \& retransmissions &
		\checkmark & balance between goodput and recovery probability & packet
		delivery time minimization & $\times$ & $\times$  \\
		\hline
		MPMTP\cite{MPMTP}  & application layer & $\times$ &Raptor codes & \checkmark & goodput maximization & block arrival time
		minimization & $\times$ & $\times$  \\
		\hline
		CMT-VR\cite{CMT-VR}   & transport layer & inflexible model & Raptor codes \&
		retransmissions & \checkmark & utility maximization& utility maximization &
		\checkmark & $\times$ \\
		\hline
		Dante    &application layer & proposed
		flexible model & RS code & \checkmark & hierarchical
		protection and distortion minimization & hierarchical protection & \checkmark & \checkmark \\
		\hline
	\end{tabular}
	\caption{MAIN DIFFERENCE OF THIS WORK WITH THE EXISTING WORKS}
	\label{}
\end{table*}
